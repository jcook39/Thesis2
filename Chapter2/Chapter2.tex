\chapter{Single Body Dynamics Vehicle Model}\label{ch:RSBD}
Tracked ground vehicles are superior to articulated steered ones in tasks where maintaining mobility is the highest priority. This has motivated work on light weight, robotic platforms and heavy class vehicles used in off-road applications. The discussion here will focus only on heavy class vehicles. This is most appropriate for the application of interest as the models for the terrain-vehicle interaction starkly contrast each other and the two vehicle types have very different powertrains. 

To date, all of the heavy class, tracked vehicle literature has come from the terramechanics field. Most of this work has focused on developing high fidelity models for the mechanics of the terrain-track interaction parametrized by vehicle and terrain properties. There is a vastness of literature that proposes different empirical models from an array of authors. The most useful and extensively validated have been summarized by Wong \cite{Wong2008}. Other authors have used these results to derive numerical models to simulate vehicle trajectories \cite{kitano1976theoretical,kitano1977analysis,murakami1992mathematical,le1997estimation,ferretti1999modelling,ahmad2000path}. All of these models use a driver or track speed as the model input or use a torque input without any inclusion of an engine and powertrain model in simulations. The short coming here is that there is no insight as to whether or not the trajectory is physically possible or in a reachable part of the state space. One author has derived a full multi-body dynamics tracked vehicle model including a powertrain of a M113 military armed-carrier vehicle \cite{rubinstein2004detailed}. However, there is no investigation into the vehicle dynamics under load while the fidelity of other vehicle subsystems is beyond what is needed for guidance, navigation, and control of unmanned ground vehicles.

This chapter discusses the modeling approaches used to derive a single body dynamics model for a skid-steered tractor. The motivation for the approaches used comes from the fact that control of autonomous tractors under load must account for various dynamic effects without excessive computational burden. To account for all the dynamics, the model must include the following salient features: driver inputs, an engine and powertrain model, Wong-Bekker terramechanics theory, payload effects on the drawbar load, and the vehicle rigid body dynamics. These features are all coupled together as the payload puts limitations on vehicle capabilities due to the power-load relationship at the engine output. This chapter is outlined as follows. The tractor and its towing configuration will be discussed as it pertains to model development. Then, the differential equations governing the rigid body dynamics for the tractor states will be derived. The following section will review the Wong-Bekker terramechanics theory being used for simulations along with a previous study that provides a relationship between payload mass and the load it places at the drawbar of the vehicle. Thereafter, the engine and powertrain models will be outlined. The final section will show two sets of open-loop simulation results. The first set of results will highlight the model's capability to predict immobilization. The second set show how the payload effects the power-load relationship at the engine output.
\import{Chapter2/}{Tractor_Towing_Configuration.tex}
\import{Chapter2/}{Rigid_Single_Body_Dynamics.tex}
\import{Chapter2/}{Terramechanics.tex}
\import{Chapter2/}{Power-Train.tex}
\import{Chapter2/}{Simulation_Results_RSBD}
\import{Chapter2/}{Terrain_Vehicle_Model_Validation}