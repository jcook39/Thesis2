\chapter{Traction Control Mode} \label{ch:TCM}
Chapter \ref{ch:LFCM} discussed the architecture used in the leader-follower control mode. This mode addresses operation needs when tractors have little to no risk of immobilization. However, if an unmanned tractor suddenly encounters softer terrain where the snow-ice surface can not handle an arbitrary shear load, the vehicle must transition into the traction control mode. This second control mode will no longer use a manned leader's throttle and gear selection but instead choose its own in order to maintain mobility so that the vehicle does not dig itself into the terrain from using excessive driver torque.

The effectiveness of any traction control implementation depends on the capabilities to assess terrain characteristics or parameters. Related work in terrain characterization has focused on lighter weight, wheeled platforms \cite{iagnemma2004online,hutangkabodee2006soil,ojeda2006terrain,ray2009estimationTerrainForcesParameters,ray2009estimationNetTraction}. The equations governing models for the terrain-interaction for this vehicle class are different and suffer inaccuracies as light weight vehicles are much more susceptible to disturbances caused by terrain unevenness and irregularity. The closest related work for traction control has been one author who derived an Extended Kalman Filter (EKF) for a 1.5 tonne tracked vehicle to obtain slip estimates for 2D plane motion for traction control \cite{le1997estimation}. However, no information regarding assumptions about measurement noise are stated, driver speeds are used as model inputs, and simulation results show maximum slip values of $1\%$ in the trajectory.

The method used for terrain characterization used here is most similar to \cite{ray2009estimationTerrainForcesParameters}. Here an EKF is used to estimate the vehicle state and vehicle-terrain forces. The state and force estimates form slip-force pairs that are compared with a set of slip-force curve models. These models are computed \textit{apriori} from sets of terrain parameters defined in the terramechanics literature \cite{Wong2008}. The most likely model is selected recursively using Bayesian Inference. This will be referred to as a multiple model estimation method (MME). 

This approach is modified for a heavy class tracked vehicle pulling a payload. Here a process model that limits the tractor to longitudinal trajectories for the Kalman Filter is used. This process model is linear with respect the tractor state and track-terrain forces. Therefore a Discrete Time Kalman Filter (DTKF) can be used and allows the filter to run with fixed gain which significantly decreases the required computational effort. Furthermore, in \cite{ray2009estimationTerrainForcesParameters}, the \textit{apriori} hypothesis set is limited to 21 sets of terrain parameters from data found in \cite{Wong2008} and the shear deformation modulus $K$ is fixed for each terrain type of sand, clay, and snow. These fixed $K$ values can significantly degrade the performance of a traction control architecture in practice as this parameter has a significant effect on the slip point that produces maximum net traction. Here, we assume bounds on the terrain parameter space and discretize within the bounds to evaluate a larger hypothesis set of 320 models. This allows for better estimates of maximum net traction and the slip set point that produces it. Furthermore, a traction feed-forward, feedback PIDF controller and gear selection controller are implemented to allow tractors to maintain mobility in soft-terrain. It should be noted that Bekker-Wong terramechanics theory is widely accepted and validated for heavy-class tracked vehicles in steady, longitudinal motion and only breaks down in high speed transients which are not present for the application of interest \cite{Wong2008}. This makes the MME approach more viable for implementation than smaller, rover platforms. 

This chapter is organized as followers. The terramechancis theory introduced in Ch. \ref{ch:RSBD} is reviewed and augmented to include slip-sinakge effects in section \ref{s:Terramechanics_Revisited}. Section \ref{s:DTKF} modifies the single body dynamics model derived in Ch. \ref{ch:RSBD} to develop a new process model and derives the DTKF used to make state-force estimates. Simulation results including the augmented terramechanics and DTKF state-force estimates are shown to demonstrate the effect on mobility predictions and the accuracy of estimates. Section \ref{s:MME} shows how results from the DTKF are used to characterize the terrain in real time to estimate the maximum net traction and the slip point that produces it using Bayesian Inference for MME. Section \ref{s:TC} outlines the structure of the feed-forward, feedback PIDF traction controller. Section \ref{s:TCMA} reviews all the elements of the traction control mode and the criteria for transitioning in and out of the this mode. Section \ref{s:SRTC} show simulation results for the traction control mode for 5 tractors to demonstrate its effectiveness for maintaining tractor mobility. 

\import{Chapter6/}{Terramechanics_Revisited}
\import{Chapter6/}{State_Force_Estimation_DTKF}
\import{Chapter6/}{MME_Bayesian_Inference}
\import{Chapter6/}{Traction_Controller}
\import{Chapter6/}{Traction_Control_Mode_Architecture}
\import{Chapter6/}{Simulations_Results}