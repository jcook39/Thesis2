\chapter{Winch Control Mode}

Chapter \ref{ch:TCM} discussed the architecture of the traction control mode and demonstrated its effectiveness in simulation results. However, if there is no domain of slip ratios that provide positive $F_{net,DB}$ values, this mode will not prevent the tractor from being immobilized. The only thing that can be done is to reduce the drawbar load on the vehicle which requires a towing winch to release the payload. An approach to automate winch use will be proposed in this chapter and will be referred to as the Winch Control Mode. The hybrid rigid body dynamics and hydraulic actuator modeling for simulating this mode was discussed in Ch. \ref{ch:RMBDW}.

The approach for this control mode is different than traditional ones as in the last chapter where parameters and states are estimated and then input references are computed for closed-loop control. Instead, inputs are tried iteratively and based on the response new inputs are computed via logic at a low bandwidth of 4 Hz. This is done since the traction control mode and winch control mode can be active simultaneously and running two interacting, higher bandwidth closed-loop controllers could cause chatter and/or instability.

There are three stages during winch use. The first stage activates the winch control mode. At this point, the tractor has determined that the traction control mode alone can not maintain vehicle mobility and lets out the payload. The second stage requires braking of the hydraulic winch so as to bring the sled back into motion and reduce the rate at which the cable is being let out until it reaches zero. This is done by slowing increasing the hydraulic pressure of the brake valve until it is determined that the vehicle can no longer handle additional drawbar load. If the vehicle's drawbar has been overloaded, the hydraulic pressure is reduced and then the pressure is slowly increased again at a reduced rate. Once the winch has been stopped, the third stage of winch use pulls the payload back in. 

This chapter will be broken up into four sections. The first three will go over the details of the approach used for each stage: detecting mobility issues and letting out the winch, braking the winch to initiate sled motion and cease cable lose, and pulling the payload back in. The last section will show two different types of simulation results. The first type will simulate the red tractor under the same conditions in the previous chapter where the traction control mode architecture did not accuratly estimate the $i_{pk}$ value and became immobilzed. This time however, once the winch becomes active the tractor will be able to traverse the terrain. The second type of simulation result will show four tractor traversing soft terrain with no positive $F_{net,DB}$ values across the $i$ domain using the automation approach proposed in this chapter. 

\import{Chapter7/}{Winch_Detection}
\import{Chapter7/}{Winch_Initializing_Payload_Recovery}
\import{Chapter7/}{Full_Winch_Recovery}
\import{Chapter7/}{Simulation_Results}