\chapter{Multi Body Dynamics Vehicle Model}\label{ch:RMBD}
The single body dynamics terrain-vehicle model presented in Ch. \ref{ch:RSBD} is validated for longitudinal motion. The model has low computational cost and will prove to be useful in later chapters for the second and third control modes. However, the single body model assumes the vehicle drawbar is loaded only in one direction for all trajectories. Since the vehicle of interest is a skid-steered tractor pulling a sled as its payload, the extent of skidding and slipping can be quite severe during turning manuevers and could lead to payload instability. These effects must be accounted for when developing controllers in the leader-follower control mode which motivates the development of a more comprehensive model that accounts for both the vehicle and payload bodies in the dynamics. The multi-body model presented in this chapter builds off the single body model in Ch. \ref{ch:RSBD}. The same powertrain and terrain-vehicle coupling is used but the assumption of unidirectional loading on the tractor's drawbar is eliminated. This is necessary for having a model to predict accurate trajectories during turning maneuvers. In addition, this provides a way to ensure controllers drive the tractor-sled vehicle in a stable and kinematicly admissible manner.

Most of the modeling done for stability analysis and control in vehicle-trailer combinations has been for articulated steered vehicles in low-slip conditions with light loads \cite{kang2007vehicle}. Under these conditions, linearized kinematic models are sufficient for analytical approximations for vehicle and trailer stability. However, the combination of a skid-steered, tracked vehicle pulling a sled in off-road conditions greatly increases the extent of skidding and slipping since there are no cornering tire forces to constrain the kinematics of the vehicle and towed load from lateral movement. Therefore, these approaches do not readily apply to this application. Only by developing a comprehensive model and simulating the closed-loop dynamics can be stability be ensured for unmanned tractors.  
\import{Chapter3/}{Lagrange_Model_Derivation}
\import{Chapter3/}{Simulation_Results}